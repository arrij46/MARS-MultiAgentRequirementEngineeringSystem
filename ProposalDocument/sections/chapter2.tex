\chapter{Project Description}
\label{ch:description}
\section{Scope}
\vspace{-1.06cm}
The scope of this project is to design and implement an AI-assisted system that supports requirement elicitation, quality assessment, refinement, and partial automation of the Software Requirement Specification (SRS) document. The system will accept user input in text form or document upload, limited to English, and provide a chatbot-based interface to enable natural and guided elicitation of requirements. It will generate context-aware clarifying questions to resolve ambiguities and missing details through iterative interaction with users. Drafted requirements will be automatically analyzed for duplication, conflicts, and classification, with the system suggesting corrections where possible and escalating complex issues to users through human-in-the-loop support. The system will further automate the creation of a partially complete SRS, covering introduction, audience, scope, product perspective, functional and non-functional requirements, glossary, and user classes. Users will be able to select templates, refine sections, and add new requirements with chatbot assistance. Additionally, finalized requirements will be transformed into structured user stories to support agile development processes. However, the scope excludes functionalities such as software coding, testing, graphical modeling, or generating diagrams. It will not support multiple projects simultaneously or collaborative editing, and users will retain full responsibility for final review and validation. The system also does not address hardware or architectural constraints that may impact performance. By focusing specifically on elicitation, refinement, and structured documentation, the project establishes clear boundaries and ensures that users receive AI-driven support for requirement engineering without extending into development or operational activities.

\section{Modules}
The proposed system is divided into the following key modules, each addressing a specific phase of the requirement engineering process and collectively contributing to the automated generation of high-quality software requirement specifications.

\subsection{Conversational Requirement Elicitation \& Drafting Module}
This module enables users to interact with the web application through an AI-assisted conversational chatbot designed for requirement elicitation. The system engages users in iterative dialogue, asking clarifying questions to resolve ambiguities and ensure completeness. User-provided inputs are then organized into a preliminary structured format that serves as the foundation for subsequent refinement and validation.  
\begin{enumerate} 
    \item Provides an interactive conversational assistant to guide the elicitation process.  
    \item Generates clarifying questions to capture precise and complete requirements. 
    \item Accepts both free text inputs and structured lists of requirements.  
    \item Produces an initial structured draft of requirements for further refinement.  
\end{enumerate}

\subsection{Requirement Quality Assurance Module}
This module ensures the clarity, consistency, and accuracy of drafted requirements by applying automated quality checks. The system identifies issues such as ambiguity, lack of atomicity, duplication, and logical conflicts, and suggests corrective actions to improve the overall reliability of the requirements.  
\begin{enumerate}
    \item Enforces atomicity by breaking down complex requirements into simpler, precise statements.  
    \item Detects duplicate and conflicting requirements for resolution.  
    \item Highlights conflicting or incomplete requirements for user clarification.  
\end{enumerate}

\subsection{Requirement Refinement \& Categorization Module}
This module further enhances validated requirements by refining their structure and categorizing them into appropriate groups. Functional and non-functional requirements are distinctly identified to support systematic documentation and traceability. Any conflicts that cannot be automatically resolved are logged for later human verification.  
\begin{enumerate}
    \item Automatically classifies requirements into functional and non-functional categories.    
    \item Provides a categorized view of requirements for improved organization and accessibility.  
    \item Refines requirement phrasing and maintains a record of unresolved conflicts for human review.  
\end{enumerate}

\subsection{User Story Generation Module}
This module bridges the gap between requirements engineering and agile development by converting validated requirements into structured user stories. Each user story is enriched with essential elements such as a title, description, and acceptance criteria, ensuring clarity and alignment with agile practices.  
\begin{enumerate}
    \item Generates user stories with structured elements (title, description, acceptance criteria).  
    \item Cost and risk analysis of user stories using the project input elicited during the elicitation phase.
    \item Prioritization of user stories to aid in sprint planning and resource allocation.  
\end{enumerate}

\subsection{SRS Document Generation \& Editing Module}
This module automates the creation of a Software Requirement Specification (SRS) document, integrating all validated requirements, user stories, and glossary terms into a coherent format. The system adheres to IEEE standards while also supporting user-provided templates for customization. Users can refine the document interactively with chatbot assistance and export it in multiple formats for practical use.  
\begin{enumerate}
    \item Assembles an editable SRS document including introduction, scope, functional and non-functional requirements, and glossary.  
    \item Adheres to IEEE standards while supporting customizable templates provided by the user.  
    \item Offers real-time chatbot-assisted editing and refinement of the document.  
    \item Exports the finalized SRS into multiple formats such as PDF, DOCX, and Markdown.  
\end{enumerate}
\section{System Architecture}
\begin{figure}[h!]
\centering
\begin{tikzpicture}[node distance=1.5cm]

% (Minimal fallback styles in case they're not in the preamble)
\tikzstyle{block} = [rectangle, draw, fill=blue!15, text centered, rounded corners, minimum height=2em, minimum width=6cm]
\tikzstyle{line} = [draw, thick, -latex']
\tikzstyle{side} = [rectangle, draw, fill=green!15, text centered, rounded corners, minimum height=2em, minimum width=3cm]

% Main pipeline
\node [block] (input) {User Input (Text / Document Upload)};
\node [block, below of=input] (elicitation) {Conversational Requirement Elicitation \& Drafting};
\node [block, below of=elicitation] (quality) {Requirement Quality Assurance};
\node [block, below of=quality] (refinement) {Requirement Refinement \& Categorization};
\node [block, below of=refinement] (userstories) {User Story Generation};
\node [block, below of=userstories] (srs) {SRS Document Generation \& Editing};
\node [block, below of=srs] (output) {Final Outputs (SRS Docs, User Stories)};

% Arrows (pipes) with labels
\path [line] (input) -- node[midway,right]{Raw Requirements} (elicitation);
\path [line] (elicitation) -- node[midway,right]{Drafted Requirements} (quality);
\path [line] (quality) -- node[midway,right]{Validated Requirements} (refinement);
\path [line] (refinement) -- node[midway,right]{Refined \& Categorized Requirements} (userstories);
\path [line] (userstories) -- node[midway,right]{User Stories + Requirements} (srs);
\path [line] (srs) -- node[midway,right]{Finalized SRS / Export} (output);

% Side components (placed using explicit coordinates relative to pipeline nodes)
\node [side] (frontend) at ($(elicitation.east)+(4cm,0)$) {Frontend (React.js Web UI)};
\node [side] (backend)  at ($(quality.east)+(4cm,0)$)     {Backend (Python/Flask or FastAPI)};
\node [side] (ai)       at ($(refinement.east)+(4cm,0)$)  {AI/NLP Engine (NLTK, PyTorch)};
\node [side] (db)       at ($(userstories.east)+(4cm,0)$) {Database (PostgreSQL/MySQL)};

% Connect side components
\path [line] (frontend.west) -- (elicitation.east);
\path [line] (backend.west)  -- (quality.east);
\path [line] (ai.west)       -- (refinement.east);
\path [line] (db.west)       -- (userstories.east);

\end{tikzpicture}
\caption{Pipe-and-Filter System Architecture of MARS}
\label{fig:architecture}
\end{figure}

The architecture of the proposed system follows a \textbf{pipe-and-filter model}, where each stage of the requirement engineering process is represented as a modular component connected in sequence. User input flows through a series of processing stages, with the output of one stage serving as the input for the next. This ensures that requirements are progressively refined, validated, and transformed into structured artifacts such as user stories and SRS documents.

At the entry point, users interact with the system through a \textbf{frontend web application}, implemented in React.js, which provides both text-based input and document upload options. The frontend connects to the \textbf{backend services}, developed in Python (Flask/FastAPI/Django), responsible for orchestrating workflows and exposing APIs. Core intelligence is handled by the \textbf{AI/NLP engine}, which leverages libraries such as NLTK and PyTorch to perform requirement elicitation, ambiguity detection, conflict resolution, and classification.

Validated requirements and generated artifacts are stored in the \textbf{database layer} (PostgreSQL/MySQL or MongoDB for unstructured storage), ensuring persistence and traceability. The pipeline includes modules for conversational elicitation, quality assurance, refinement, user story generation, and SRS assembly, each functioning as independent but interconnected processing blocks. Additionally, the system integrates a \textbf{document processing component}, using tools like \texttt{python-docx} and Pandoc, to support export into multiple formats such as PDF, DOCX, and Markdown.

This modular architecture promotes scalability, maintainability, and flexibility. For example, improvements to the AI models or integration of new document templates can be performed without affecting the entire system. By adopting the pipe-and-filter design, the architecture ensures that requirements flow in a structured manner from raw input to finalized documentation while maintaining opportunities for human-in-the-loop interaction and feedback.
\section{Tools and Technologies}

\subsection*{Frontend (Web Application Interface)}
\begin{itemize}
    \item React.js  
    \item HTML5, CSS3, JavaScript 
    \item Bootstrap  
\end{itemize}

\subsection*{Backend (Application Logic and APIs)}
\begin{itemize}
    \item Python (Flask/FastAPI/Django)
    \item Node.js (optional)
\end{itemize}

\subsection*{AI/ML and NLP Technologies}
\begin{itemize}
    \item NLTK 
    \item PyTorch 
\end{itemize}

\subsection*{Database and Storage}
\begin{itemize}
    \item PostgreSQL / MySQL 
    \item MongoDB (optional) 
    \item File Storage 
\end{itemize}

\subsection*{Document Processing \& Export}
\begin{itemize}
    \item python-docx 
    \item Pandoc 
\end{itemize}

\section{Work Division}
For each module and respective Feature, assign responsibility to a team member

\begin{table}[!ht]
\caption{Table 1}
\centering
\small
\begin{tabular}{|p{3cm}|p{2cm}|p{9cm}|}
\hline
\textbf{Name} & \textbf{Registration} & \textbf{Responsibility / Module / Featureure} \\ \hline
Mr. Zubair & 22i-2475 & Module 1- Feature 1, Module 2- Feature 1, Module 3- Feature 1, Module 4- Feature 2, Module 5- Feature 3  \\ \hline  
Miss. Arrij & 22i-0755 & Module 1- Feature 2-3, Module 2- Feature 1, Module 3- Feature 2, Module 4- Feature 1, Module 5- Feature 4  \\ \hline  
Miss. Hamna & 22i-1098 & Module 1- Feature 4, Module 2- Feature 3, Module 3- Feature 3, Module 4- Feature 3, Module 5- Feature 2 \\ \hline
Team & - & Module 1- Feature 1, Module 2- Feature 2, Module 5- Feature 1   \\ \hline  
\end{tabular}
\end{table}

\section{Timeline}

\begin{table}[!ht]
\caption{Project Iteration Plan} 
\begin{tabular}{|c|c|p{9cm}|} \hline
\textbf{Iteration \#} & \textbf{Time Frame} & \textbf{Tasks / Modules} \\ \hline
01 & Sept–Oct & Development of Conflict and Duplicate Detection Model, including data preprocessing and validation. \\ \hline
02 & Nov–Dec  & Implementation of Ambiguity Detection, Requirement Classification, and Refinement mechanisms. \\ \hline
03 & Jan–Feb  & Conversational Requirement Elicitation and Automated User Story Generation. \\ \hline
04 & Mar–Apr  & Integration of SRS Editor, Chatbot Interface, Requirement Filtering, and Preliminary Cost–Risk Analysis. \\ \hline
\end{tabular}
\end{table}