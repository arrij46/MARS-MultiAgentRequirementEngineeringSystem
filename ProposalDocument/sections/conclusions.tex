\chapter{Conclusions and Future Work}

\section{Conclusion}
The proposed project outlines the design and implementation of an AI-assisted system that enhances the requirements engineering process by focusing on elicitation, refinement, quality assurance, and structured documentation.
Through its modular architecture, the system provides an interactive chatbot interface for iterative requirement gathering, applies automated checks for consistency and accuracy, and transforms validated requirements into both structured user stories and a partially automated SRS document aligned with IEEE standards. 
The integration of natural language processing techniques with document generation tools ensures that users benefit from AI-driven support while maintaining human oversight for final validation.
By clearly defining the system’s scope, excluding aspects such as coding, testing, or architectural design, the project ensures a targeted approach that addresses some of the most pressing challenges in requirement engineering. The systematic work division and structured timeline further highlight the feasibility and collaborative nature of the development process. 
Overall, this project contributes toward bridging the gap between requirement elicitation and agile development practices, offering a practical, intelligent, and user-friendly solution that streamlines the creation of reliable and high-quality software requirement specifications.

\section{Future Work}
While the proposed system provides a structured and AI-assisted approach to requirement elicitation and documentation, there remain several opportunities for further enhancement. A key direction for future work is the development of a multi-perspective requirement elicitation engine capable of capturing requirements from diverse stakeholder viewpoints, ensuring broader coverage and reducing bias in the gathered specifications. 
Another avenue involves incorporating adaptive learning mechanisms, such as Cross Talk Instance Memory, to enable the system to learn from previous interactions and refine its responses over time, thereby improving accuracy and contextual relevance.
In addition, expanding interoperability with widely used software project management tools, such as GitHub, spreadsheets, and Trello, can significantly improve the system’s practical utility by enabling seamless integration into existing workflows. These extensions would not only broaden the system’s applicability but also strengthen its role in supporting collaborative, agile, and large-scale software development projects.

