\chapter{Introduction}
\label{sec:introduction}
\section{Problem Statement}
\vspace{-0.5cm} 
Software development projects often face challenges not due to technical shortcomings, but because of poorly defined requirements that compromise the foundation of the system. Requirements engineering (RE) is widely recognized as one of the most critical yet error-prone phases of the development lifecycle. Conventional approaches rely heavily on manual effort and the availability of experts, making the process time-consuming, inconsistent, and prone to human error. This creates barriers for non-specialists, reduces efficiency, and increases the likelihood of costly mistakes.
The Standish Group’s CHAOS Report highlights this risk, showing that a clear statement of requirements contributes 13\% to project success, while 12.3\% of challenged projects are caused by incomplete or unclear specifications \cite{1}. These figures underline how even small lapses in requirement clarity can trigger significant delays, rework, and in many cases, outright project failure. Traditional RE practices often produce Software Requirements Specification (SRS) documents that suffer from ambiguity, inconsistency, and conflicts, leading to miscommunication among stakeholders.
Research by IBM underscores the importance of accuracy at the outset, showing that fixing a defect in production can cost up to 100 times more than correcting it during the requirements phase \cite{2}.
Ultimately, the recurring problems of incompleteness, ambiguity, and inconsistency in requirements undermine software quality and project success. These challenges highlight the urgent need for an intelligent, systematic solution that enhances requirement clarity, reduces errors, and streamlines the transition to development.

\section{Motivation}
The motivation for this project arises from the critical role that requirements engineering plays in determining software success and the persistent challenges associated with it. Studies indicate that around 56\% of software defects originate in the requirements phase \cite{3}, showing how errors at this stage create the most damaging ripple effects throughout development. These figures highlight the urgency of improving practices that ensure clarity, completeness, and consistency in requirements.
At the same time, the requirements management market is projected to reach USD 3.2 billion by 2033 \cite{4}, with AI integration emerging as a key driver of growth. This trend reflects a broader industry recognition that traditional approaches are insufficient and that intelligent, automated support is increasingly essential. The combination of high technical risk and growing market demand provides strong motivation to explore solutions that enhance requirements engineering practices.
Ultimately, the driving motivation behind this project is the opportunity to reduce costly downstream errors, improve project outcomes, and contribute to the evolving landscape of requirements engineering by addressing one of the most persistent pain points in software development.

\section{Problem Solution}
The proposed software, MARS (Multi-Agent Requirements Engineering System), is an AI-driven assistant designed to directly address the recurring challenges identified in the requirements engineering process. Traditional practices often result in incomplete, ambiguous, or conflicting specifications, which increase rework, miscommunication, and project failures. To overcome these limitations, MARS integrates automation, reasoning, and conversational interaction into a unified platform, ensuring requirements are captured, validated, and organized with greater accuracy and efficiency.
The first objective of MARS is to engage users in a conversational elicitation process, enabling both technical and non-technical stakeholders to express requirements without the need for specialized expertise. Second, the system performs automated quality assurance, detecting ambiguity, conflicts, duplicates, and non-atomic requirements at the earliest stage, thereby reducing costly downstream errors. Third, it ensures proper categorization of requirements into functional and non-functional types, aligning them with established software engineering practices.
Beyond validation and classification, MARS contributes to agile development workflows by generating user stories from refined requirements, making them directly actionable for development teams. Another key goal is to produce editable, standards-compliant Software Requirements Specification (SRS) documents, which include core sections such as Introduction, Scope, Functional and Non-Functional Requirements, and Glossary. At the same time, MARS allows the inclusion of elements, added and customized directly by users through the integrated chatbot, ensuring the document remains adaptable to project-specific needs.
By meeting these objectives, MARS minimizes the reliance on manual effort, reduces the expertise barrier for high-quality requirements engineering, and accelerates the process of moving from stakeholder intentions to structured specifications. The ultimate goal is to embed quality at the requirements stage, where errors are most costly, while making professional-grade requirements engineering more accessible, consistent, and adaptable across diverse project environments.

\section{Stake Holders}
\begin{enumerate}
\item Clients / End-Users – Provide the requirements and validate that the final SRS reflects their needs.
\item Business Analysts / Requirements Engineers – Use the system to elicit, refine, and structure requirements into a high-quality specification.
\item Project Managers – Depend on accurate, conflict-free requirements to plan resources, timelines, and deliverables.
\item Software Developers – Rely on the generated user stories and categorized requirements to guide implementation.
\item Quality Assurance (QA) Teams – Use the clarified requirements to design effective test cases and ensure product correctness.
\end{enumerate}
