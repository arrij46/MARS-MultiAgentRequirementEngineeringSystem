\chapter{Introduction}
\label{sec:introduction}
\section{Introduction to the Domain}
Requirements Engineering (RE) is a fundamental area of software engineering that focuses on identifying, analyzing, documenting, and managing the requirements of a system. It establishes a shared understanding between stakeholders and developers, ensuring that the final product aligns with user needs and organizational objectives. By providing a structured process for capturing both functional and non-functional requirements, RE plays a critical role in improving communication, guiding design and implementation, and supporting successful software delivery.
\section{Problem Statement}
\vspace{-0.5cm} 
Software development projects often fail not due to technical shortcomings, but because of poorly defined requirements that compromise the foundation of the system. Requirements Engineering (RE), while critical, remains one of the most error-prone phases of the development lifecycle. Conventional approaches depend heavily on manual effort and expert availability, making the process time-consuming, inconsistent, and prone to human error. This reliance creates barriers for non-specialists, reduces efficiency, and increases the likelihood of costly mistakes.  

The Standish Group’s CHAOS Report highlights this risk, showing that a clear statement of requirements contributes 13\% to project success, while 12.3\% of challenged projects stem from incomplete or unclear specifications \cite{1}. Research by IBM further underscores that fixing a defect in production can cost up to 100 times more than correcting it during the requirements phase \cite{2}. These figures demonstrate how even small lapses in requirement clarity can trigger significant delays, rework, or outright project failure.  

\textbf{The recurring problems of incompleteness, ambiguity, and inconsistency in requirements continue to undermine software quality and project success.}  
\textbf{This project specifically addresses the need for an intelligent, systematic solution that enhances requirement clarity, reduces errors, and streamlines the transition to development.}
\section{Motivation}
The motivation for this project lies in the central role of Requirements Engineering in shaping the success of software development. Since requirements define what is to be built, improving the way they are captured, refined, and managed directly enhances the efficiency and reliability of the entire development process. By ensuring that requirements are clear, consistent, and well-structured from the outset, teams can reduce rework, shorten development cycles, and improve collaboration among stakeholders.  

This project seeks to leverage intelligent support to make requirements engineering more accessible and systematic, lowering the reliance on extensive manual effort and specialized expertise. In doing so, it aims to enable software teams to focus more on building solutions rather than resolving misunderstandings or correcting errors later in the lifecycle. With the software industry increasingly embracing automation and AI-driven tools, there is strong motivation to explore how these technologies can optimize RE practices and bring measurable improvements to project outcomes.  

\textbf{The motivation behind this project is to make requirements engineering smarter, faster, and less error-prone, thereby strengthening the entire software development process.}  
\textbf{By embedding quality at the requirements stage, the project aims to improve efficiency, reduce risks, and deliver more dependable outcomes across diverse software projects.}
\section{Existing Solutions}
Several AI-assisted tools and platforms have been introduced to support requirements engineering and related activities. These systems aim to automate aspects of requirements authoring and management, but their focus is often limited to specific functions rather than providing a comprehensive, quality-driven approach. Table \ref{tab:existing-solutions} provides an overview of notable solutions currently in the market.

Although these systems demonstrate the potential of AI to support requirements-related tasks, they reveal notable shortcomings. \textbf{Current tools lack intelligent requirement suggestions, rigorous quality-checking, conflict detection, and version control for evolving documentation.} These gaps highlight the need for a more comprehensive solution that combines elicitation, validation, refinement, and documentation within a single intelligent framework.
\begin{table}[h!]
\centering
\begin{tabular}{|p{4cm}|p{4cm}|p{4cm}|}
\hline
\textbf{System} & \textbf{Key Features} & \textbf{Identified Gaps} \\ \hline
ScopeMaster & Checks grammar, clarity, complexity, and testability of requirements. & Does not provide requirement suggestions; lacks cross-checking for duplicates/conflicts; no version control. \\ \hline
Azure Copilot4DevOps & Generates user stories, tasks, and test cases to support DevOps workflows. & Focused on task generation rather than quality assurance; no rigorous requirement validation or version control. For Azure workflows only.\\ \hline
WriteMyPRD & Quickly generates Software Requirements Specification (SRS) documents with minimal input. & Prioritizes speed over quality; lacks ambiguity/conflict checks; minimal elicitation assistance. No versioning support.\\ \hline
Beam’s PRD AI & Produces fast and structured Product Requirement Documents (PRDs). & Limited to document generation; no requirement suggestions or conflict detection; lacks version control. \\ \hline
\end{tabular}
\caption{Comparison of existing AI-assisted requirements tools and their limitations}
\label{tab:existing-solutions}
\end{table}
\section{Problem Solution}
The proposed software, MARS (Multi-Agent Requirements Engineering System), is an AI-driven assistant designed to directly address the recurring challenges identified in the requirements engineering process. Traditional practices often result in incomplete, ambiguous, or conflicting specifications, which increase rework, miscommunication, and project failures. To overcome these limitations, MARS integrates automation, reasoning, and conversational interaction into a unified platform, ensuring requirements are captured, validated, and organized with greater accuracy and efficiency.

The key objectives and features of MARS are outlined as follows:
\begin{enumerate}
    \item \textbf{Conversational Requirement Elicitation:}  
    Engages users in a chatbot-based elicitation process, enabling both technical and non-technical stakeholders to express requirements naturally, without requiring specialized expertise.

    \item \textbf{Automated Requirement Quality Assurance:}  
    Performs automated checks for ambiguity, conflicts, duplicates, and non-atomic requirements at the earliest stage, thereby reducing costly downstream errors.

    \item \textbf{Requirement Classification and Refinement:}  
    Ensures proper categorization of requirements into functional and non-functional types, aligning them with established software engineering practices, while refining their phrasing and clarity.

    \item \textbf{User Story Generation for Agile Workflows:}  
    Converts refined requirements into actionable user stories enriched with titles, descriptions, and acceptance criteria, making them directly usable for development teams.

    \item \textbf{SRS Document Generation and Editing:}  
    Produces editable, standards-compliant Software Requirements Specification (SRS) documents, covering sections such as Introduction, Scope, Functional and Non-Functional Requirements, and Glossary.  
    The system also supports chatbot-assisted customization and integration of project-specific details.

    \item \textbf{Accessibility and Adaptability:}  
    Minimizes reliance on manual effort, lowers the expertise barrier for high-quality requirements engineering, and ensures that documents remain adaptable to diverse project environments.
\end{enumerate}

By meeting these objectives, MARS embeds quality in the requirements stage, where errors are most costly, while making professional-grade requirements engineering more accessible, consistent, and efficient.
\section{StakeHolders}
\begin{enumerate}
\item Clients / End-Users – Provide the requirements and validate that the final SRS reflects their needs.
\item Business Analysts / Requirements Engineers – Use the system to elicit, refine, and structure requirements into a high-quality specification.
\item Project Managers – Depend on accurate, conflict-free requirements to plan resources, timelines, and deliverables.
\item Software Developers – Rely on the generated user stories and categorized requirements to guide implementation.
\item Quality Assurance (QA) Teams – Use the clarified requirements to design effective test cases and ensure product correctness.
\end{enumerate}
